\chapter*{Zusammenfassung}

Nachfolgende Projektarbeit befasst sich der Evaluieren von Frameworks für die hybride und Cross-Platform Anwendungsentwicklung. Zunächst werden vergangene Arbeiten bezüglich oben genannter Thematik analysiert. Darauf folgend findet eine Erklärung der Begriffe hybride und Cross-Platform-Entwicklung statt und die in dieser Arbeit relevanten Plattformen werden vorgestellt. Zudem gibt es eine Einführung in die Funktionsweisen der Sensoren, deren Verwendung in dieser Arbeit getestet wird. Hierauf folgt die Anforderungsanalyse, welche neben der Problemanalyse die funktionalen und nichtfunktionalen Anforderungen an die Evaluation aufzeigt.
\\
\\
In der darauf folgenden Marktanalyse werden mit verschiedenen Werkzeugen die 3 aktuell für die mobile Anwendungsentwicklung relevantesten Frameworks ermittelt. Diese werden anschließend kurz vorgestellt. Im Anschluss wird der Vorgang zur Aufstellung der Bewertungskriterien, nach denen die ausgewählten Frameworks evaluiert werden, erläutert und die Bewertungskriterien werden einzeln erklärt. 
\\
\\
Nach der Aufstellung der Bewertungskriterien folgt die Planung der Funktionstestanwendung, mit deren Hilfe die ausgewählten Frameworks bezüglich der Verwendung verschiedener Funktionalitäten getestet werden sollen. Diese Anwendung wird im Weiteren in einer objektorientierten Analyse und Design anhand der zugehörigen UML-Diagramme definiert und vorgestellt. Anschließend wird die Implementierung der Funktionstestanwendung unter Verwendung der 3 ausgewählten Frameworks erläutert. In der nachfolgenden Auswertung werden die 3 ausgewählten Frameworks gemäß der Erfahrungen bei der Entwicklung und weiterer Recherchen anhand der zuvor aufgestellten Bewertungskriterien evaluiert. 
\\
\\
Zum Schluss wird ein zusammenfassender Vergleich der evaluierten Frameworks gegeben und eine Empfehlung ausgesprochen. Es folgt ein Ausblick auf weiterführende Arbeiten.  