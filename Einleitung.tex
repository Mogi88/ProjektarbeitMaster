\chapter{Einleitung} \label{Einleitung}

Über die vergangenen Jahre haben es Smartphones unter Endverbrauchern zu enormer Beliebtheit geschafft und sind mittlerweile überall im Alltag in Gebrauch\footcite{RealChallengesMobileApp}. Mit dem Smartphone werden Urlaube geplant, Kinokarten gekauft, Rechnungen bezahlt. Wir benutzen das Smartphone und seine Anwendungen um mit Dienstleistern zu kommunizieren\footcite{ChallOppMobAppDev}. Diese Allgegenwart von Smartphones hat entsprechend die Aufmerksamkeit von Software Entwicklern auf sich gezogen. Stand 2013 befanden sich etwa 800.000 mobile Anwendungen in Apple's AppStore, 650.000 im Android Store, 120.000 im Windows Marketplace und 100.000 in Blackberry's AppWorld\footcite{RealChallengesMobileApp}. Die rasante Entwicklung sieht man, wenn man diese Zahlen mit dem Stand 2016 vergleicht: etwa 2.000.000 Anwendungen befinden sich in Apple's AppStore und in Androids Google Play sogar mehr als 2.560.000\footcite{StatistikApps}. Wie in oben genannter Statistik bereits erkennbar, gibt es mehrere Handelsplattformen mit jeweils einem eigenen Betriebssystem, Entwicklerwerkzeuge und Bibliotheken, wie zum Beispiel Android, iOS, Windows Phone oder BlackBerry. Fast jährlich erscheinen für diese Betriebssysteme Major Releases\footcite{ChallOppMobAppDev}. Diese Fragmentierung der Plattformen und Standards führen auf der einen Seite zu verstärktem Wettkampf, was sich positiv auf Fortschritt und Weiterentwicklung auswirkt. Auf der anderen Seite ist dies eine Barriere in der Entwicklung von Inhalten und Services, was die Benutzer an eine spezifische Technologie bindet oder Entwicklerunternehmen einen Mehraufwand bereitet, um ihre Services auf mehreren Plattformen zu anzubieten\footcite{ChallMobAppDev}. 
\\
\\
Die Charakteristiken von mobilen Anwendungen und die Erwartungen der Benutzer führen zu folgender Basisherausforderung in der mobilen Anwendungsentwicklung: 'Liefere schnell aus und reagiere schnell auf Feedback', und dies in einem nicht endenden Zyklus. Jedes Unternehmen, dessen mobile Anwendungen 1-Stern Bewertungen und Kommentare mit Adjektiven  wie 'verwirrend', 'unvollständig' oder 'langsam' bekommt, wird auf dem Markt verlieren, wenn diese Kritiken nicht ernst genommen und schnell Verbesserungen implementiert werden\footcite{ChallOppMobAppDev}. 
\\
\\
Die 3 großen Kategorien in der mobilen Anwendungsentwicklung sind nativ, Web-basiert und hybrid. Native Anwendungen sind für ein spezifisches Betriebssystem entwickelt und nutzen dessen Schnittstellen und Bibliotheken direkt. Sie müssen für jedes Betriebssystem auf dem sie laufen sollen separat entwickelt werden. Web-basierte Anwendungen laufen in einem Web Browser, der auf dem Smartphone installiert sein muss. Hybride Anwendungen sind 'native-wrapped' Web-Anwendungen. Bei nativen Anwendungen können, im Gegensatz zu Web-basierten Anwendungen, alle nativen Funktionen des Endgeräts genutzt werden, wie zum Beispiel Kameras und Sensoren\footcite{ChallOppMobAppDev}. 
\\
\\
Hybride und Cross-Plattform Anwendungen versuchen den Vorteil der Plattformunabhängigkeit Web-basierter Anwendungen mit dem Vorteil des Nutzens nativer Funktionen von nativen Anwendungen zu vereinen. Diese Form von Anwendungsentwicklung erzielt weltweit immer mehr Popularität aufgrund der Tatsache, dass sich ihr Code auf mehreren Plattformen kompilieren lässt. Entwicklungswerkzeuge für hybride und Cross-Plattform Anwendungsentwicklung basieren meist auf Programmiersprachen aus der Webentwicklung wie zum Beispiel HyperText Markup Language (HTML), JavaScript oder Cascading Style Sheets (CSS). Dazu kommt dann eine Art Wrapper-Code für die Zugriffe auf die nativen Schnittstellen (APIs). Auf diese Weise können dann auch die nativen Funktionen wie Kameras und Sensoren angesprochen werden. Diese Form von Anwendungsentwicklung verspricht eine Reduzierung der Entwicklungskosten bei neuen Anwendungen, was sie für Entwicklerunternehmen attraktiv macht\footcite{ComparisonCrossPlatMobDevTools}. Die Arbeit von Palmieri, Singh und Cicchetti\footcite{ComparisonCrossPlatMobDevTools} hat allerdings auch ergeben, dass ein gravierender Nachteil hybrider Anwendungen im Gegensatz zu nativen Anwendungen die Performance ist. Aus diesem Grund beschäftigt sich diese Arbeit mit dem Vergleich verschiedener Frameworks, die für Entwicklung hybrider und Cross-Plattform Anwendungen angeboten werden. Ein besonderes Augenmerk wird dabei auch auf die Performance bei der Nutzung nativer Funktionen im Vergleich zur nativen Anwendung gelegt.