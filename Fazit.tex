\chapter{Zusammenfassung und Ausblick}

Wie in der Auswertung (Kapitel \ref{Auswertung}) zu lesen und in der Auswertungsmatrix zu sehen ist, liegen die Stärken und Schwächen der einzelnen Frameworks in unterschiedlichen Bereichen. Betrachtet man die Abschnitte 'Kosten und Lizenz', 'Support und Community' und 'Tests' kombiniert, so schlägt sich das umfangreichere Angebot an Hersteller-Support und Testunterstützung der Frameworks Xamarin und Ionic merkbar in den Kosten nieder. Das kostenfreie Framework React Native schneidet entsprechend am schlechtesten in diesen Punkten ab. Wobei hier noch zu erwähnen ist, dass die Kosten beim Ionic Framework für vergleichbare Leistungen noch höher sind als beim Xamarin Framework. Beim Xamarin Framework ist man hingegen gezwungen bei den kostenpflichtigen Varianten die IDE 'Visual Studio' zu kaufen und man hat nur die Möglichkeit Abonnements abzuschließen, welche über 2 Jahre dauern. Umgerechnet auf den Monat ergeben sich zwar trotz 'Visual Studio'-Lizenz geringere Kosten als bei Ionic, dafür genießt man bei Ionic ein monatliches Kündigungsrecht. 
\\
\\
Die Unterschiede im Bereich Community-Support ist dem Alter des jeweiligen Frameworks geschuldet. Dies ist an den recherchierbaren Themen in zum Beispiel der 'Stackoverflow'-Community (Kapitel \ref{AuswsuppComm}) und an der Google Trends Recherche (siehe Kapitel \ref{Marktanalyse}) zu erkennen. Bei Ionic ist hier zusätzlich noch der Sprung zur Version Ionic 2 zu berücksichtigen, die viele Änderungen mit sich brachte (vgl. Kapitel \ref{AuswEntwicklung}). Können sich die Frameworks React Native und Ionic 2 mit Cordova weiter durchsetzen, ist hier über Zeit eine Angleichung zu erwarten. Aufgrund des vergleichbar jungen Entwicklungsstands des React Native Frameworks und den daraus resultierenden regelmäßigen gravierenden Änderungen am Framework ist zudem noch anzumerken, dass Code-Beispiele, die ein paar Monate alt sind, oft mit der aktuellen Version nicht mehr kompilierbar sind. Schwierigkeiten bei der Verwendung von Beispiel-Code blieben allerdings auch wie in Kapitel \ref{AuswEntwicklung} beschrieben bei den Frameworks Xamarin und Ionic nicht aus.
\\
\\
Was die Installation und Benutzbarkeit betrifft, konnten mit allen Frameworks ähn-liche Punktzahlen erreicht werden (siehe Kapitel \ref{AuswEntwicklung}). Hier bietet die Nutzung der IDE 'Visual Studio' bei der Verwendung des Xamarin Frameworks sowohl Vorteile als auch Nachteile gegenüber der Programmierung im Editor bei Ionic und React Native (siehe ebenfalls Kapitel \ref{AuswEntwicklung}). Der umständliche Umgang mit benötigten Komponenten in der IDE 'Visual Studio' steht hier den Annehmlichkeiten von Autovervollständigung und Code-Generierung gegenüber. Die Installation und die Benutzbarkeit der IDE Xamarin Studio für die Entwicklung mit Xamarin auf einem Apple Computer konnte im Rahmen dieser Arbeit nicht evaluiert werden. Bei der kürzeren benötigten Einarbeitungszeit beim Xamarin Framework spielt natürlich auch der größere Umfang an Informationen in Form von unter anderem Beispiel-Code und Community-Beiträgen eine Rolle (vgl Kapitel \ref{AuswsuppComm}). Zudem dürften sich Android- bzw. Java-Ent-wickler schneller in der Implementierung mit C\# zurechtfinden, die sehr nah an der nativen Entwicklung ist, als mit der Entwicklung mit JSX und TypeScript. Für Web-Entwickler gilt dies entsprechend eher umgekehrt. Gemeinsamkeiten und Unterschiede zu der nativen iOS Entwicklung mit der Programmiersprache Swift konnten in dieser Arbeit nicht ermittelt werden, da aufgrund der Schwerpunktsetzung keine native iOS Entwicklung vorgenommen wurde. Der Anteil an plattformübergreifenden Code ist bei der Entwicklung mit Ionic und React Native größer als bei Xamarin. Dieser Vorteil schrumpft jedoch, wenn man sich bei der Entwicklung mit Xamarin für Xamarin.Forms entscheidet anstelle der Nutzung nativer UI-Elemente. Ein Vorteil der Verwendung von Plugins bei Ionic und React Native ist, dass Zugriffe auf native Funktionalitäten nicht selbst implementiert werden müssen. Dies gilt allerdings nur solange, wie Funktionalitäten genutzt werden, für die bereits fertige Plugins existieren. Andernfalls ist die Implementierung eines Plugins für eine gewünschte Funktionalität aufwendiger als die direkte Umsetzung des nativen Zugriffs im Code der Anwendung wie bei Xamarin. Ob die verwendeten Plugins auch die getesteten Funktionalitäten für eine iOS Anwendung bereitstellen, oder ob hierfür weitere Plugins gesucht, bzw. implementiert werden müssen, müsste in Folgearbeiten untersucht werden.
\\
\\
Bei den Möglichkeiten zur Gestaltung der grafischen Benutzeroberfläche sieht es so aus, dass bei Xamarin direkt die nativen Elemente von Android verwendet werden können. Auf diese Weise ist eine Xamarin-Anwendung optisch nicht von einer nativen zu unterscheiden. Dieser Vorteil geht allerdings Hand in Hand mit dem oben beschriebenen Nachteil des geringeren Anteils an plattformübergreifenden Code. Bei der Verwendung von Xamarin.Forms hingegen ist die Benutzeroberfläche von Ionic und auch React Native Anwendungen näher an der nativen Benutzeroberfläche als die der mit Xamarin.Forms implementierten Xamarin-Anwendung. Der Umgang mit dem GUI-Design ist aufgrund der ausführlichen Dokumentation (siehe Kapitel \ref{AuswGUI}) mit dem Ionic Framework am einfachsten. Der GUI-Designer des Xamarin Frameworks ist leider für die Verwendung nativer, dem 'Material Design' entsprechender Elemente nicht benutzbar (siehe ebenfalls Kapitel \ref{AuswGUI}). Der GUI-Designer des Xamarin Studios für die Entwicklung von iOS Anwendungen wurde im Rahmen dieser Arbeit aufgrund der Schwerpunktsetzung nicht evaluiert. 
\\
\\
Was den Umfang der nutzbaren nativen Funktionalitäten betrifft, so sind mit allen getesteten Frameworks theoretisch alle überprüften Funktionalitäten umsetzbar. Die größten Unterschiede sind hier beim Aufwand zu finden. So sind mit dem Xamarin Framework alle getesteten Funktionalitäten mit gleichem eher geringem Aufwand umsetzbar (siehe Kapitel \ref{AuswFunktion}). Auch die Art der Umsetzung unterscheidet sich nur geringfügig. Sowohl bei Ionic als auch bei React Native vergrößert sich der Aufwand für die Umsetzung einer Funktionalität schlagartig, sobald für diese kein fertiges Plugin zur Verfügung steht oder nur schwer zu finden ist, da es in der Dokumentation nicht erwähnt wird.
\\
\\
Im Bereich Performance und Speichergröße konnte das Framework Xamarin bei den getesteten Funktionsaufrufen zwar wie in Kapitel \ref{AuswPerf} beschrieben den Spitzenplatz bei der Performance erzielen, dafür ist die Speichergröße eines APKs einer Xamarin Anwendung schnell um ein vielfaches größer als die der Konkurrenz. Was die Performance betrifft wiesen die Frameworks auch unterschiedliche Stärken und Schwächen bei den getesteten Funktionen auf. So schnitt zum Beispiel die React Native Anwendung bei der Verwendung der Kamera zwar schlecht, beim Laden von Dateien und Lesen von Sensordaten dafür sehr gut ab. Der Rahmen, in dem die Performance-Möglichkeiten der Frameworks getestet wurden, war allerdings sehr begrenzt. Hier würden sich noch weitere, differenziertere Tests empfehlen.
\\
\\
In den Bereichen 'Interoperabilität' und 'Sicherheit' sind wie in Kapitel \ref{AuswInterop} zu lesen ist, keine nennenswerten Unterschiede zwischen den 3 Frameworks Xamarin, Ionic und React Native aufgefallen.
\\
\\
Einige der oben genannten Schwachpunkte der Frameworks React Native und Ionic 2 könnten auf Alter und Entwicklungsstand der beiden Frameworks in der jeweils aktuellen Version zurückzuführen sein. Eine Bestätigung oder Verwerfen dieser Annahme ist zu diesem Zeitpunkt allerdings nicht möglich. Welches der Frameworks für langfristige Planungen am empfehlenswertesten ist, lässt sich aus eben diesem Grund heute nicht beantworten. Die Entwicklung der Frameworks müsste weiter beobachtet und erneut evaluiert werden. Zudem wurden im Zuge dieser Arbeit nur Basisfunktionalitäten und Zugriffsmöglichkeiten mit den Frameworks getestet. Wie sich Anwendungen, die mit den unterschiedlichen Frameworks geschrieben wurden, in Lastsituationen wie zum Beispiel dem Anzeigen hochauflösender Bildsequenzen verhalten, kann daher zu diesem Zeitpunkt ebenfalls nicht beurteilt werden. Auf dem jetzigen Stand lässt sich allerdings folgende Empfehlung aussprechen: Handelt es sich bei der zu entwickelnden Anwendung um eine Anwendung in der viele unterschiedliche UI-Elemente zum Einsatz kommen sollen, so empfiehlt sich aufgrund der Aufwandsminimierung die Verwendung des Frameworks Ionic. Ist die zu entwickelnde Anwendung hingegen sehr Business-Logik-lastig so empfiehlt sich die Verwendung des Frameworks Xamarin. 
\\
\\
Aufgrund der Schwerpunktsetzung dieser Arbeit auf die Plattform Android fehlen viele Vergleiche der Frameworks bezüglich der Benutzung für iOS Anwendungen. In nachfolgenden Arbeiten könnte unter anderem der Aufwand der Übersetzung einer mit den getesteten Frameworks für Android entwickelten Anwendung nach iOS evaluiert werden. Zu Beginn einer Folgearbeit ist jedoch zu empfehlen alle zu untersuchenden Frameworks auf zwischenzeitliche Änderungen und Weiterentwicklungen hin zu prüfen. 