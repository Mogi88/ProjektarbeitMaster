\chapter{Auswertung}

Im folgenden Kapitel wird die Auswertung anhand der in Kapitel (X) aufgestellten Bewertungskriterien beschrieben. Zu den einzelnen Kategorien (siehe Kapitel (X)) wird der jeweilige ausgefüllte Abschnitt der Bewertungsmatrix dargestellt und erläutert. 

\section{Kosten und Lizenz}

Nachfolgende Abbildung (X) zeigt den Abschnitt 'Kosten und Lizenz' der ausgefüllten Bewertungsmatrix.

\begin{figure}[h]
	\centering
	\includegraphics[width=1\textwidth]{Bilder/Auswertung_KostenLizenz.PNG}
	\caption{Bewertungsmatrix Kategorie Kosten und Lizenz}
	\label{fig:AuswKostLiz}
\end{figure}

Das Xamarin Framework gibt es in einer kostenlosen Community-Version und in den kostenpflichtigen Versionen 'Professional' und 'Enterprise'. Die kostenlose Community-Version gibt es für Windows inklusive einer Community-Version von Visual Studio und für Apple mit der IDE 'Xamarin Studio'. Die Preise der kostenpflichtigen Versionen richten sich nach den Preisen der entsprechenden Version der IDE Visual Studio. Die Professional-Version kann für 499\$ ohne Subscription oder für 1199\$ mit Subscription erworben werden. Eine Subscription hält 2 Jahre, ein Auffrischen danach kostet 799\$. Die Enterprise-Version kann nur mit Subscription erworben werden. Sie kostet 5999\$ und ein Auffrischen nach Ablauf der 2 Jahre kostet 2569\$. Einen technischer Support  erhält man mit jeder Subscription, das bedeutet ab umgerechnet 50\$ pro Monat. E-Mail-Support erhalten alle Business- und Enterprise-Kunden. Zusätzlich zu allen Modellen kann die Xamarin Test-Cloud ab einen monatlichen Preis von 99\$ genutzt werden.
\\
\\
Für das Framework React Native gibt es keine kostenpflichtigen Versionen. 
\\
\\
Für das Framework Cordova gibt es ebenfalls keine kostenpflichtigen Modelle. Das Framework Ionic bietet dagegen folgende Varianten an: eine kostenlose Community-Version und die kostenpflichtigen Versionen 'Indie', 'Business' und 'Enterprise'. Im Gegensatz zu den kostenpflichtigen Modellen bei Xamarin, fallen bei den Modellen von Ionic monatliche Subscription-Gebühren an. Die Indie-Variante kostet 25\$ pro Monat und die Business-Variante kostet 99\$ pro Monat. Für die Enterprise-Version gibt es einen individuellen Preis auf Anfrage. Ab der Business-Variante erhält man E-Mail-Support, das bedeutet ab 99\$ pro Monat. Möchte man die Ionic Test Cloud nutzen, so kostet dies 20\$ pro Monat und Anwendung.  

\section{Support und Community}

\section{Entwicklung}

Die ausgefüllte Bewertungsmatrix für die Kategorie 'Entwicklung' ist in nachfolgender Abbildung (X) dargestellt.

\begin{figure}[h]
	\centering
	\includegraphics[width=1\textwidth]{Bilder/Auswertung_Entwicklung.PNG}
	\caption{Bewertungsmatrix Kategorie Entwicklung}
	\label{fig:AuswEntw}
\end{figure}

Hat man bereits ein Visual Studio installiert und möchte Xamarin integrieren, so ist dies nicht ohne Mehraufwand möglich. Einfacher ist es, Visual Studio direkt zusammen mit Xamarin zu installieren. Hierzu muss nur der Installer von der Xamarin Homepage heruntergeladen und ausgeführt werden. Alles Notwendige für die Entwicklung mit Xamarin ist bei der Installation von Visual Studio bereits vorausgewählt. Der Xamarin Installer installiert zusätzlich noch weitere benötigte Komponenten, wie ein Android SDK und NDK. Es muss keine weitere Software separat beschafft werden. Startet man nach der erfolgreichen Installation die IDE Visual Studio, so kann direkt mit der Entwicklung mit Xamarin gestartet werden, indem ein neues Projekt 'Android App' angelegt wird. Anders als bei der nativen Entwicklung mit Android Studio gibt es hier allerdings keine Design-Vorauswahl bei der Projekterstellung wie zum Beispiel einen \textit{NavigationDrawer} für die Navigation. Dafür gibt es sogenannte 'Pre Built Apps', das sind beispielhafte fertige Anwendungen, wie unter anderem Shopping- oder CRM-Anwendungen. 
\\
Auf der Xamarin Homepage findet sich einiges an Beispielcode zu verschiedensten Funktionalitäten. Viele dieser Beispielprojekte sind allerdings nicht direkt kompilierbar. Es ist immer aufwendiges recherchieren notwendig, welche Verweise, Components oder Einstellungen im Projekt angepasst werden müssen, da diese Informationen nicht mit angegeben sind. Für die Verwendung von zum Beispiel nativen UI-Elementen müssen sogenannte 'Components' installiert und dem Projekt hinzugefügt werden. Welche 'Components' für ein gewünschtes Feature benötigt werden muss selbst recherchiert werden. Bei der Auswahl dieser 'Components' gibt es oft Schwierigkeiten, da manche 'Components' zwingend in einer übereinstimmenden Version installiert sein müssen, um Versionskonflikte zu vermeiden. Welche 'Components' übereinstimmende Versionen haben müssen, kann nur durch Recherche oder Probieren ermittelt werden. Hierbei muss zusätzlich darauf geachtet werden, dass eventuell nicht in allen Versionen das gewünschte Feature enthalten ist.
\\
Xamarin Anwendungen werden, wie schon in Kapitel (X) erwähnt, mit der objektorientierten Programmiersprache C\# entwickelt. Da C\# vom Aufbau her Java sehr ähnlich ist, ist es für einen Android- oder Java-Entwickler keine große Umstellung. Die Projektstruktur einer Android-Anwendung mit Xamarin ist identisch mit der nativen Projektstruktur. Die Layout-Dateien können mit wenigen Anpassungen von einer nativen Android-Anwendung übernommen werden. Mit diesen Voraussetzungen ist das Portieren einer nativen Android-Anwendung nach Xamarin unkompliziert. Die API für Zugriffe auf Hardware wie Sensoren und Speicher wirkt wie eine 1 zu 1 Umsetzung der Android API. Viele Klassen heißen gleich oder sehr ähnlich. Auch die Verwendung deckt sich mit der nativen. Dies macht es zwar jedem Android-Entwickler leicht, sich in Xamarin einzuarbeiten, jedoch bedingt dies zugleich mit sich, dass eben diese Code-Abschnitte nicht plattformübergreifend nutzbar sind. Ein höherer Anteil an plattformübergreifenden Code kann mit der Nutzung von Xamarin.Forms erreicht werden, wobei dann auf native UI-Elemente verzichtet werden muss. Zum Umfang der von Xamarin zur Verfügung gestellten Bibliotheken ist zu sagen, das im Rahmen dieser Arbeit keine Android-Funktionalität gefunden wurden, die nicht mit Xamarin umsetzbar waren. 
\\
\\



\section{Hersteller}

\section{OS-Versionen}

\section{Funktionsumfang}

\section{GUI-Design}

\section{Interoperabilität}

\section{Tests}

\section{Performance}

\section{Programmiersprache}

\section{Sicherheit}
